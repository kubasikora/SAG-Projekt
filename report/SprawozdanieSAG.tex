\documentclass{article}
\pdfpagewidth=8.5in
\pdfpageheight=11in

\usepackage{SAGreport}
% Use the postscript times font!
\usepackage{times}
\usepackage{soul}
\usepackage{url}
\usepackage{xcolor}
\usepackage{polski}
\usepackage[polish]{babel}
\usepackage[utf8]{inputenc}
\usepackage[T1]{fontenc}
\usepackage[utf8]{luainputenc}
\usepackage[hidelinks]{hyperref}
\usepackage[utf8]{inputenc}
\usepackage{caption}
\usepackage{indentfirst}
\usepackage{graphicx}
\usepackage{amsmath}
\usepackage{siunitx}
\usepackage{booktabs}
\usepackage{subfig}
\usepackage{listings}
\lstset{
	language=bash,
	basicstyle=\ttfamily
}
\urlstyle{same}
	
\title{Systemy agentowe\\ Planowanie pracy w fabryce}

\author{
Maria Konieczka, Alicja Poturała, Jakub Sikora
\affiliations
numery albumów: 283410, 283415, 283418 \\
\emails
maria.konieczka.stud@pw.edu.pl, alicja.poturala.stud@pw.edu.pl, jakub.sikora2.stud@pw.edu.pl
}

\newcommand{\todo}[1]{\textcolor{blue}{\textbf{TO DO:} #1}}
\newcommand{\opracowanyartykul}[1]{\textcolor{teal}{\textbf{Artykuł:} #1\\}}

\begin{document}
\maketitle

\section{Wprowadzenie}
\label{sec:wprowadzenie}
\todo{napisać o problemie, o co w nim chodzi, dlaczego trzeba go rozwiazac, moze cos o licznosci zbioru ktory trzeba przeszukac}

\section{Zarys algorytmu}
\label{sec:algorytm}
Przedstawiony algorytm negocjacji został zaczerpnięty z~pracy \emph{Production sequencing as negotiation}~\cite{wooldridge1996production}. W~ninejszej pracy, autorzy proponują rozwiązanie problemu planowania sekwencji pracy w~fabryce, zamieniając problem na zadanie negocjacji w~systemie agentowym.

Pomysł zamiany problemu na zadanie negocjacji opiera się na założeniu, że fabrykę można podzielić na kilka komórek. Z~każdą komórką związana jest funkcja kosztu, która opisuje pewien koszt wytworzenia elementu. Koszt produkcji związany jest z~zarówno wykonaniem samego elementu, jak i~zależy od wcześniej wykonywanego elementu. Przykładowo, stacja lakiernicza, która ostatnio pracowała z~czerwonym lakierem, w~celu wykonania zielonego elementu, musi zmienić lakier, wyczyścić dysze oraz wykonać szereg innych prac, z~którymi można utożsamić pewien koszt.

W~systemie agentowym, negocjacja sekwencji polega na rozmowie pomiędzy agentami, reprezentującymi poszczególne komórki fabryki, podczas której każdy z~agentów próbuje zmaksymalizować swoją funkcję użyteczności, czyli \emph{de facto} zminimalizować swój koszt. Koszt związany z~samą produkcją jest stały dla każdej zadanej sekwencji, dlatego też minimalizowany będzie koszt zmian typu produktu.

Podejście agentowe ma wiele zalet. W~pierwszej kolejności, takie rozwiązanie jest modularne, co pozwala na łatwe rozszerzanie, dołączanie i~wyłączanie komponentów. Rozbicie problemu na podproblemy, pozwala znacząco uprościć problem optymalizacji oraz potencjalnie przyspieszyć proces szukania rozwiązania, poprzez zrównoleglenie obliczeń.

\begin{figure}
    \centering
    \includegraphics[width=0.8\columnwidth]{figures/SAG-Negotiation.pdf}
    \caption{Schemat algorytmu negocjacji z podziałem na stany}
    \label{fig:abstract-negotiation-fsm}
\end{figure}

Na rysunku~\ref{fig:abstract-negotiation-fsm} przedstawiony został ogólny schemat algorytmu negocjacji. W~stanie $s_{0}$ dokonywana jest inicjalizacja algorytmu. W~każdym agencie, ustawiany jest licznik rund $t$ na wartość początkową równą $1$ oraz zbiór propozycji przedstawionych $T_{i}$ jako zbiór pusty. Następnie, każdy z~agentów ustawiany jest w~stan \textit{Active} oraz przeliczany jest zbiór $b_{p_{i}}$. Ostatnim krokiem jest niedeterministyczny wybór pierwszej propozycji $\sigma$.

Po inicjalizacji, algorytm przechodzi w~stan $s_{1}$, w~którym każdy aktywny agent wykłada na stół swoją propozycję $\sigma$, równocześnie dołączając ją do swojego zbioru propozycji przedstawionych $T_{i}$. Następnie, agenci rozmawiając ze sobą szukają takiej propozycji, która odpowiada wszystkim aktywnym agentom. Jeżeli taka propozycja zostanie znaleziona, następuje przejście do stanu terminalnego $s_{2}$. W~przeciwnym przypadku, należy znaleźć agenta, który powinien odpuścić w~danej rundzie negocjacji. Aby znaleźć takiego agenta, negocjacje przechodzą do stanu $s_{3}$.

W~stanie $s_{3}$, każdy z~aktywnych agentów oblicza swoje ryzyko. Ryzyko jest liczbową wartością jak dużo agent może stracić, innymi słowami jak bardzo może wzrosnąć koszt agenta jeżeli w~danej rundzie odpuści. Z~wszystkich agentów, wybierani jest zbiór agentów $g$, dla których ryzyko jest najmniejsze. Po określeniu zbioru $g$, negocjacje przechodzą do stanu $s_{4}$.

W~stanie $s_{4}$ następuje etap szukania kompromisu. Propozycję kompromisową zaproponować mogą tylko agenci ze zbioru $g$, określonego w~poprzednim kroku. Agenci poszukują \emph{prawdziwego kompromisu}, czyli takiej propozycji która jest co najmniej tak dobra dla każdego agenta jak ta którą zaproponował w~danej rundzie oraz lepsza dla co najmniej jednego innego agenta. Proponowane rozwiązanie nie może zwiększać kosztu globalnego całej fabryki oraz nie może być już wcześniej zaproponowane przez agenta szukającego kompromisu. Jeżeli agent znajdzie propozycję spełniającą podane obostrzenia, to zaproponuje ją w~następnej rundzie negocjacji. Negocjacje przechodzą do stanu $s_{1}$ i~rozpoczyna się kolejna iteracja. Jeżeli nie to agent przechodzi do stanu $s_{5}$.

Obostrzenia dotyczące \emph{prawdziwego kompromisu} są dosyć spore, dlatego też może zdarzyć się że agent nie znajdzie takiej sekwencji. W~takim przypadku, w~stanie $s_{5}$ agent ponownie przelicza swój zbiór propozycji $b_{p_{i}}$. Jeżeli jest on niepusty, to w~następnej iteracji zaproponuje jedną z~sekwencji z~nowego zbioru $b_{p_{i}}$, wybraną w~sposób niedeterministyczny.

Może dojść do sytuacji że nowy zbiór $b_{p_{i}}$ będzie zbiorem pustym, co świadczy o~wyczerpaniu możliwych propozycji. W~takim przypadku, agent przechodzi w~stan nieaktywny i opuszcza stół negocjacyjny. Na rysunku~\ref{fig:abstract-negotiation-fsm} zostało to przedstawione jako przejście do stanu $s_{6}$. Przejście do tego stanu znaczy że agent wyczerpał swoje możliwości negocjacyjne i~zgadza się na wszystkie inne propozycje, przedstawione w~dalszych etapach.


\section{Architektura i~implementacja rozwiązania}
\label{sec:architektura}
W~celu implementacji przedstawionego algorytmu negocjacji, wykorzystaliśmy język programowania ogólnego przeznaczenia Python w~wersji 3.6 oraz bibliotekę \texttt{SPADE - Smart Python Agent Development Environment}. Biblioteka ta upraszcza proces tworzenia agentów oraz obsługuje komunikację między nimi, która oparta jest o~protokół XMPP.

\section{Mechanizmy odpornościowe}
\label{sec:recovery}
\todo{opisać mechanizm recovery na który się zdecydujemy, manager kontroluje agenty, jak on padnie to ktorys z nich przejmuje zadanie}

\section{Skalowanie rozwiązania}
\label{sec:skalowanie}
Najtrudniejszym i najbardziej wymagającym obliczeniowo elementem w \texttt{NegotiateFSM} jest wyznaczanie zbioru B0. Zbiór B0' składa się z wszystkich sekwencji dla danego agenta, które pozwalają na zminimalizowanie jego kosztu. Zbiór B0 zawiera natomiast te elementy, które mają najmniejszy sumaryczny koszt dla całej fabryki spośród zbioru B0'.

Aby wyznaczyć zbiór B0 należy więc dla wszystkich sekwencji z B0' wysłać wiadomości do wszystkich współpracowników. W całym systemie, na tym etapie negocjacji wysyłanych jest $(n-1)\sum_{i \in Ag}|B0'_i|$ wiadomości, gdzie $n$ to liczba agentów \texttt{FactoryAgent} w systemie. 

Wbrew pozorom, podczas przeprowadzania testów, okazało się, że wielkość zbioru B0' odgrywa większą rolę na szybkość wyznaczania rozwiązania, niż liczba agentów. Wielkość zbioru B0' dla każdego agenta zależy wykładniczo od liczności nierozróżnialnych dla danego agenta produktów do wyprodukowania. Wyznaczenie B0' sprowadza się do obliczenia permutacji wewnątrz nierozróżnialnych zbiorów a następnie wyznaczenia wszystkich możliwych kombinacji połączeń tych zbiorów. 

Liczba nierozróżnialnych dla danego agenta produktów zależy od liczby typów produktów zleconych do wykonania. Dla agenta, który odpowiada za cechę, która może mieć trzy wartości, w przypadku zlecenia 12 różnych produktów, w najkorzystniejszym przypadku, liczność zbioru B0' wynosi $4!\cdot 4!\cdot 4!\cdot 6$ czyli $\num{82944}$, natomiast dla 11 różnych produktów $3!\cdot 4!\cdot 4!\cdot 6$ czyli $\num{20736}$.

Z tego powodu postanowiliśmy ograniczyć liczbę typów do 9. Wykorzystanie algorytmu jest nadal opłacalne, gdyż przeszukiwany zbiór rozwiązań jest mniejszy o kilka rzędów wielkości niż $9!$. Jeśli chodzi o liczbę wiadomości wysyłanych w systemie, to w ogólności zależy ona od kwadratu liczby agentów (dokładnie $n \cdot (n-1)$)


\lstset{
	tabsize=2,
	breaklines=true,
	basicstyle=\tiny,
}
\section{Testy}
\label{sec:testy}
\todo{na jakim kompoie jakie wyniki ile czasu komentarz ze niekoniecznie optymalne}

\section{Podsumowanie i wnioski}
\label{sec:podsumowanie}
Podczas projektu udało się zaimplementować algorytm pozwalający na rozwiązanie problemu minimalizacji kosztu globalnego za pomocą zespołu agentów minimalizujących lokalne koszty. Znalezione rozwiązanie nie jest globalnie optymalne jednak jest na tyle rozsądne aby uzyskać egalitarny kompromis pomiędzy kosztami poszczególnych komórek fabryki.

Osiągnięcie najmniejszego możliwego kosztu całej fabryki mogłoby wiązać się z sytuacją w której kilka komórek ma bardzo duży koszt, podczas gdy pozostałe mają najniższy z możliwych. Egalitarne podejście do problemu pozwala wyrównać koszty poszczególnych komórek, co w realnym świecie wiązałoby się z równomiernym zużyciem elementów fabryki.

Algorytm zaproponowany przez autorów~\cite{wooldridge1996production} nie sprawdza się dla dużych zbiorów B0'. Już dla zbioru 10 nierozróżnialnych typów aut dla agenta jego zbiór B0' przekracza $10!$ co stanowi znaczne ograniczenie zastosowania. 

Okazało się, że w przypadku mechanizmów naprawczych, znacznie większą rolę ma \texttt{ManagerAgent}, który wykrywa błąd w zachowaniu po maksymalnie 60. sekundach. W przypadku automatu stanów, obliczanie zbioru B0 dla dużych zbiorów B0' trwa dłużej, z tego powodu, by umożliwić działanie algorytmu należało zwiększyć limit.

Wykorzystanie biblioteki \texttt{SPADE} nie było najlepszą decyzją projektową. Okazało się, że nie wspiera ona tak podstawowych akcji jak na przykład resetowanie agentów i nie jest najwygodniejsza w użyciu. \texttt{SPADE} jest cały czas rozwijaną platformą, o czym może świadczyć fakt, że najnowsza wersja pochodzi z 22.05.2020r.


\bibliographystyle{acm}
\bibliography{bibliography}

\end{document}