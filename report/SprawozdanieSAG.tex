\documentclass{article}
\pdfpagewidth=8.5in
\pdfpageheight=11in

\usepackage{SAGreport}
% Use the postscript times font!
\usepackage{times}
\usepackage{soul}
\usepackage{url}
\usepackage{xcolor}
\usepackage{polski}
\usepackage[polish]{babel}
\usepackage[utf8]{inputenc}
\usepackage[T1]{fontenc}
\usepackage[utf8]{luainputenc}
\usepackage[hidelinks]{hyperref}
\usepackage[utf8]{inputenc}
\usepackage{caption}
\usepackage{indentfirst}
\usepackage{graphicx}
\usepackage{amsmath}
\usepackage{siunitx}
\usepackage{booktabs}
\usepackage{subfig}
\usepackage{listings}
\lstset{
	language=bash,
	basicstyle=\ttfamily
}
\urlstyle{same}
	
\title{Systemy agentowe\\ Planowanie pracy w fabryce}

\author{
Maria Konieczka, Alicja Poturała, Jakub Sikora
\affiliations
numery albumów: 283410, 283415, 283418 \\
\emails
maria.konieczka.stud@pw.edu.pl, alicja.poturala.stud@pw.edu.pl, jakub.sikora2.stud@pw.edu.pl
}

\newcommand{\todo}[1]{\textcolor{blue}{\textbf{TO DO:} #1}}
\newcommand{\opracowanyartykul}[1]{\textcolor{teal}{\textbf{Artykuł:} #1\\}}

\begin{document}
\maketitle

\section{Wprowadzenie}
\label{sec:wprowadzenie}
\todo{napisać o problemie, o co w nim chodzi, dlaczego trzeba go rozwiazac, moze cos o licznosci zbioru ktory trzeba przeszukac}

Problem zaplanowania pracy w fabryce można zdefiniować na wiele sposobów. Wynika to między innymi z tego, że algorytm powinien być dostosowany do konkretnego procesu i obiektu. Ponadto problem ten można zdefiniować na różnych poziomach zarządzania wytwórnią. Z jednej strony, na podstawie analizy zapotrzebowania rynku, menadżerowie powinni zdecydować o liczbie konkretnych produktów do stworzenia. Z drugiej strony, należy zaplanować w jaki sposób należy zrealizować przedstawione przez zarządców zadanie.

W ramach projektu zdecydowaliśmy się na znalezieniu pareto-optymalnego rozwiązania problemu stworzenia planu realizującego zadanie postawione przez menadżera. 

Każdego dnia pracy fabryki należy przygotować konkretną sekwencję produktów, jakie będą tworzone danego dnia. Zadaniem systemu jest zaproponowanie takiej sekwencji, która jednocześnie spełniłaby postawione przez menadżera zadanie (ile poszczególnych produktów należy wytworzyć) oraz minimalizowałaby sumaryczny koszt stworzenia obiektów. Założyliśmy, że każdy produkt opisany jest za pomocą $n$ cech. Każda z cech jest nadawana produktowi w oddzielnych komórkach linii produkcyjnej. Komórki te opisywane są za pomocą zbioru wartości cech, kosztu wytworzenia danej cechy oraz kosztu dostosowania stacji do zmiany cechy.

W ramach projektu postanowiliśmy zaimplementować algorytm planujący pracę w fabryce samochodów. Zauważyliśmy, że już w przypadku zdefiniowania tylko trzech komórek fabryki (lakiernia, spawalnia i montażownia) i określeniu dla każdej cechy zbioru trzech dopuszczalnych wartości wykorzystanie rozproszonego planowania wprowadza znaczne zmniejszenie złożoności problemu. 

Sam problem znalezienia optymalnej sekwencji w sposób klasyczny sprowadza się do przeszukania przestrzeni możliwych rozwiązań i znalezienia sekwencji minimalizującej koszty dla całej fabryki. Już dla trzech komórek fabryki i trzech dopuszczalnych wartości dla każdej z cech, zbiór możliwych rozwiązań ma wymiar $27!$. Jest to problem np trudny, dla którego konieczne jest znalezienie alternatywy do klasycznych algorytmów przeszukiwania przestrzeni. Wprowadzenie konceptu agentów, którzy będą rozwiązywali problem lokalnie (z perspektywy poszczególnych komórek fabryki) pozwoli znacznie zaoszczędzić potrzebne zasoby pamięciowe oraz obliczeniowe.

\section{Zarys algorytmu}
\label{sec:algorytm}
\todo{przedstawić algorytm z~\cite{wooldridge1996production} negocjacji pomiędzy agentami, generalnie przepisać to z artykułu plus może wstawić jakiś schemat blokowy?}

\section{Architektura i~implementacja rozwiązania}
\label{sec:architektura}
\todo{opisać architekture rozwiazania w spade~\cite{spade}, podział systemu na agenty, agenty na zachowania, zachowania na stany, dodatkowo sekwencje komunikacji pomiedzy nimi}

\section{Mechanizmy odpornościowe}
\label{sec:recovery}
\todo{opisać mechanizm recovery na który się zdecydujemy, manager kontroluje agenty, jak on padnie to ktorys z nich przejmuje zadanie}

Dla systemów agentowych kluczowym zagadnieniem jest stworzenie mechanizmów naprawczych. Należy założyć, że zarówno komunikacja pomiędzy agentami, jak i agenci jako procesy mogą ulec awarii. Sytuacje takie należy przede wszystkim wykryć oraz, jeśli jest to możliwe, należy naprawić problem. 

\subsection{Założenia}

Dla zaproponowanej przez nas architektury rozwiązania należy zdefiniować wszystkie możliwe rodzaje awarii oraz ustalić z jakimi system będzie w stanie sobie poradzić. W rzeczywistości nie jest możliwe stworzenie mechanizmów uodparniających na wszystkie rodzaje awarii. 

W przypadku wykorzystania biblioteki \texttt{SPADE} zauważone zostało, że należy wyróżnić dwa rodzaje awarii:
\begin{enumerate}
	\item awarię pojedynczego zachowania agenta,
	\item awarię całego agenta.
\end{enumerate}

Postanowiliśmy, że kluczowe z punktu widzenia algorytmu są agenty \texttt{FactoryAgent}. Z tego powodu nie zostały stworzone zachowania odpornościowe dla agenta Menadżera a sam \texttt{ManagerAgent} odgrywa istotną rolę w mechanizmach odpornościowych agentów \texttt{FactoryAgent}. 

Aby zrealizować mechanizmy odpornościowe zaprojektowano dwa zachowania: \texttt{ControlSubordinatesBehaviour} w \texttt{ManagerAgent} oraz \texttt{WatchdogBehaviour} w \texttt{FactoryAgent}.
 
Zaimplementowano mechanizmy naprawcze dla następujących sytuacji:
\begin{itemize}
	\item wykrycie zabitego zachowania w \texttt{FactoryAgent}.
	\item brak odpowiedzi na wiadomości od współpracowników.
	\item brak odpowiedzi na wiadomość od \texttt{ManagerAgent}
\end{itemize}

Ponadto system pozwala na wykrycie awarii zachowania \texttt{ControlSubordinatesBehaviour}, czyli awarii \texttt{ManagerAgent}.

\subsection{Zaimplementowane mechanizmy}
Zadaniem \texttt{ControlSubordinatesBehaviour} jest monitorowanie działania poszczególnych agentów \texttt{FactoryAgent}. Jest to zachowanie cykliczne, które z określoną częstotliwością wysyła wiadomość typu \texttt{WatchdogMessage} do zachowań \texttt{WatchdogBehaviour} wszystkich podwładnych agentów. Zachowanie to oczekuje na odpowiedź z \texttt{WorkingState}. 

Enum \texttt{WorkingState} ma zdefiniowane trzy wartości: OK, RESTARTING oraz COMPLAINT. Pierwsze dwie wysyłane są do Managera w odpowiedzi na \texttt{WatchdogMessage}, ostatnie wysyłane jest przez pozostałe zachowania agentów, jeśli nie będą one w stanie skontaktować się z współpracownikami.

Zachowanie \texttt{ControlSubordinatesBehaviour} zlicza ile razy pod rząd nie uzyskano odpowiedzi od poszczególnych podwładnych agentów lub uzyskano odpowiedź RESTARTING. Jeśli okaże się, że agenci nie odpowiadali, lub próbowali zrestartować swoje zachowania ponad dwa razy pod rząd, manager restartuje agenta.







\section{Skalowanie rozwiązania}
\label{sec:skalowanie}
Najtrudniejszym i najbardziej wymagającym obliczeniowo elementem w \texttt{NegotiateFSM} jest wyznaczanie zbioru B0. Zbiór B0' składa się z wszystkich sekwencji dla danego agenta, które pozwalają na zminimalizowanie jego kosztu. Zbiór B0 zawiera natomiast te elementy, które mają najmniejszy sumaryczny koszt dla całej fabryki spośród zbioru B0'.

Aby wyznaczyć zbiór B0 należy więc dla wszystkich sekwencji z B0' wysłać wiadomości do wszystkich współpracowników. W całym systemie, na tym etapie negocjacji wysyłanych jest $(n-1)\sum_{i \in Ag}|B0'_i|$ wiadomości, gdzie $n$ to liczba agentów \texttt{FactoryAgent} w systemie. 

Wbrew pozorom, podczas przeprowadzania testów, okazało się, że wielkość zbioru B0' odgrywa większą rolę na szybkość wyznaczania rozwiązania, niż liczba agentów. Wielkość zbioru B0' dla każdego agenta zależy wykładniczo od liczności nierozróżnialnych dla danego agenta produktów do wyprodukowania. Wyznaczenie B0' sprowadza się do obliczenia permutacji wewnątrz nierozróżnialnych zbiorów a następnie wyznaczenia wszystkich możliwych kombinacji połączeń tych zbiorów. 

Liczba nierozróżnialnych dla danego agenta produktów zależy od liczby typów produktów zleconych do wykonania. Dla agenta, który odpowiada za cechę, która może mieć trzy wartości, w przypadku zlecenia 12 różnych produktów, w najkorzystniejszym przypadku, liczność zbioru B0' wynosi $4!\cdot 4!\cdot 4!\cdot 6$ czyli $\num{82944}$, natomiast dla 11 różnych produktów $3!\cdot 4!\cdot 4!\cdot 6$ czyli $\num{20736}$.

Z tego powodu postanowiliśmy ograniczyć liczbę typów do 9. Wykorzystanie algorytmu jest nadal opłacalne, gdyż przeszukiwany zbiór rozwiązań jest mniejszy o kilka rzędów wielkości niż $9!$. Jeśli chodzi o liczbę wiadomości wysyłanych w systemie, to w ogólności zależy ona od kwadratu liczby agentów (dokładnie $n \cdot (n-1)$)


\lstset{
	tabsize=2,
	breaklines=true,
	basicstyle=\tiny,
}
\section{Testy}
\label{sec:testy}
System został przetestowany dla siedmiu przypadków. Do testowania wykorzystano maszynę wirtualną z system Ubuntu 18.04. Dla każdego z przypadków, założono, że każdy agent odpowiada za cechę, która może mieć trzy wartości.
\subsection{Testowanie dla 3 agentów}
W przypadku testowania systemu dla 3 agentów, wykorzystano plik konfiguracyjny ze zleceniem wejściowym \textit{generate\_vectore\_agent3.json}. Rozmiar zbioru B0' dla poszczególnych agentów wynosił 3456, 2880 i 1728. Agentom udało się dojść do porozumienia i wyznaczyć rozwiązanie po niecałych 29. sekundach.
\begin{lstlisting}
[manager@localhost: 2020-06-14 13:15:42.700852] DeputeBehaviour for agentc@localhost created
[manager@localhost: 2020-06-14 13:15:42.701610] Task sent to agenta@localhost
[manager@localhost: 2020-06-14 13:15:42.702028] Task sent to agentb@localhost
[manager@localhost: 2020-06-14 13:15:42.702429] Task sent to agentc@localhost
[manager@localhost: 2020-06-14 13:15:42.702685] Every task sent
...
[manager@localhost: 2020-06-14 13:16:11.452603] Got optimal sequence [26, 26, 26, 26, 6, 6, 6, 6, 6, 6, 3, 3, 3, 5, 5, 5, 5, 5, 5, 5, 5, 14, 14, 14, 14, 14, 14, 10, 10, 10, 10, 10, 18, 0, 0, 2, 2, 2] with total cost of 625
[main: 2020-06-14 13:16:11.673955] Agents finished
\end{lstlisting}

\subsection{Testowanie dla 8 agentów}
W przypadku testowania systemu dla 8 agentów, wykorzystano plik konfiguracyjny ze zleceniem wejściowym \textit{generate\_vectore\_agent8.json}. Rozmiar zbioru B0' dla poszczególnych agentów wynosił 2880, 1728, 1728, 1728, 3456, 1728, 5760 i 8640. Agentom udało się dojść do porozumienia i wyznaczyć rozwiązanie po około $\num{13,5}$ minutach.
\begin{lstlisting}
[manager@localhost: 2020-06-14 13:40:20.139839] Task sent to agenta@localhost
[manager@localhost: 2020-06-14 13:40:20.140737] Task sent to agentb@localhost
[manager@localhost: 2020-06-14 13:40:20.141691] Task sent to agentc@localhost
[manager@localhost: 2020-06-14 13:40:20.142648] Task sent to agentd@localhost
[manager@localhost: 2020-06-14 13:40:20.143593] Task sent to agente@localhost
[manager@localhost: 2020-06-14 13:40:20.145955] Task sent to agentf@localhost
[manager@localhost: 2020-06-14 13:40:20.147695] Task sent to agentg@localhost
[manager@localhost: 2020-06-14 13:40:20.148860] Task sent to agenth@localhost
[manager@localhost: 2020-06-14 13:40:20.150821] Every task sent
...
[AgentA: 2020-06-14 13:53:47.057112] Agreement found!!
[AgentA: 2020-06-14 13:53:47.057171] The end
[AgentA: 2020-06-14 13:53:47.057211] The cost 1820
[AgentA: 2020-06-14 13:53:47.057246] Winning sequence [686, 1718, 1718, 1718, 1718, 1718, 1029, 1029, 1029, 1029, 1029, 1029, 1029, 1029, 3307, 3307, 3307, 3307, 3307, 3307, 3842, 3842, 3842, 2597, 2597, 2597, 2597, 2597, 2597, 2192, 2192, 2192, 2192, 2192, 2192, 2192, 2192, 4432, 6492, 6492, 6492]
[manager@localhost: 2020-06-14 13:53:47.057812] Got optimal sequence [686, 1718, 1718, 1718, 1718, 1718, 1029, 1029, 1029, 1029, 1029, 1029, 1029, 1029, 3307, 3307, 3307, 3307, 3307, 3307, 3842, 3842, 3842, 2597, 2597, 2597, 2597, 2597, 2597, 2192, 2192, 2192, 2192, 2192, 2192, 2192, 2192, 4432, 6492, 6492, 6492] with total cost of 1820
\end{lstlisting}
\subsection{Testowanie dla 10 agentów}
W przypadku testowania systemu dla 10 agentów, wykorzystano plik konfiguracyjny ze zleceniem wejściowym \textit{generate\_vectore\_agent10.json}. Rozmiar zbioru B0' dla poszczególnych agentów wynosił 2880, 1728, 1296, 1728, 4320, 1728, 1296, 4320, 1728, 1728. Agentom udało się dojść do porozumienia i wyznaczyć rozwiązanie po około 33 minutach.
\begin{lstlisting}
[manager@localhost: 2020-06-14 13:58:58.970668] DeputeBehaviour for agentj@localhost created
[manager@localhost: 2020-06-14 13:58:58.982009] Task sent to agenta@localhost
[manager@localhost: 2020-06-14 13:58:58.996478] Task sent to agentb@localhost
[manager@localhost: 2020-06-14 13:58:59.004003] Task sent to agentc@localhost
[manager@localhost: 2020-06-14 13:58:59.018002] Task sent to agentd@localhost
[manager@localhost: 2020-06-14 13:58:59.024701] Task sent to agente@localhost
[manager@localhost: 2020-06-14 13:58:59.031039] Task sent to agentf@localhost
[manager@localhost: 2020-06-14 13:58:59.037846] Task sent to agentg@localhost
[manager@localhost: 2020-06-14 13:58:59.043934] Task sent to agenth@localhost
[manager@localhost: 2020-06-14 13:58:59.050280] Task sent to agenti@localhost
[manager@localhost: 2020-06-14 13:58:59.056439] Task sent to agentj@localhost
[manager@localhost: 2020-06-14 13:58:59.056495] Every task sent
...
[manager@localhost: 2020-06-14 14:31:54.287116] Got optimal sequence [7616, 7616, 7616, 7616, 28274, 28274, 47912, 47912, 47912, 47912, 38679, 38679, 38679, 38679, 4609, 4609, 4609, 4609, 4609, 15489, 43198, 43198, 43198, 23900, 35267, 35267, 35267, 35267, 35267, 35267, 35267] with total cost of 1760
[main: 2020-06-14 14:31:54.721081] Agents finished
\end{lstlisting}
\subsection{Testowanie mechanizmu naprawczego zachowania}
Testowanie mechanizmu naprawczego zachowanie zostało przeprowadzane dla systemu z trzema agentami. Po określonym czasie losowany został agent oraz jego zachowanie, które to następnie zostało zabijane. Restartowanie samego zachowania jest błyskawiczne, dlatego można zauważyć, że już w następnej iteracji zachowania \texttt{ControlSubordinatesBehaviour} zachowanie działa poprawnie. W przedstawionym poniżej teście wylosowany został AgentB oraz zachowanie \texttt{ComputeAgentsCost}. W logach można zauważyć, że o 14:49:33 zostało zabite zachowanie oraz, że wiadomości wysłane do tego zachowania nie zostały dostarczone. Ponad to, można zauważyć, że AgentB wykrył niedziałające zachowanie o 14:49:48. Od tego czasu pozostali agenci nie mieli problemu z nawiązaniem kontaktu, o czym może świadczyć fakt błyskawicznego przejścia do kolejnego stanu agentów AgentA i AgentC.
\begin{lstlisting}
[manager@localhost: 2020-06-14 14:49:33.063398]  killing agent 1 behaviour 
...
No behaviour matched for message: <message to="agentb@localhost" from="agenta@localhost" thread="None" metadata={'conversation-id': 'cost', 'performative': 'request', 'save': 'True'}>
[10, 10, 10, 10, 10, 14, 14, 14, 14, 14, 14, 6, 6, 6, 6, 6, 6, 2, 2, 2, 3, 3, 3, 5, 5, 5, 5, 5, 5, 5, 5, 0, 0, 26, 26, 26, 26, 18]
</message>
No behaviour matched for message: <message to="agentb@localhost" from="agentc@localhost" thread="None" metadata={'conversation-id': 'cost', 'performative': 'request', 'save': 'True'}>
[10, 10, 10, 10, 10, 6, 6, 6, 6, 6, 6, 18, 3, 3, 3, 0, 0, 14, 14, 14, 14, 14, 14, 2, 2, 2, 26, 26, 26, 26, 5, 5, 5, 5, 5, 5, 5, 5]
</message>
...
[AgentA: 2020-06-14 14:49:51.796292] Starting negotiation
...
[AgentC: 2020-06-14 14:49:52.773619] Starting negotiation
...
[manager@localhost: 2020-06-14 14:49:48.103538] Pinging agentb@localhost
...
[AgentB: 2020-06-14 14:49:48.104472] Got message
[AgentB: 2020-06-14 14:49:48.104609] Restarting
...
[manager@localhost: 2020-06-14 14:50:08.121212] Pinging agentb@localhost
[manager@localhost: 2020-06-14 14:50:08.122911] Ping responded!
\end{lstlisting}
\subsection{Testowanie mechanizmu naprawczego w przypadku zatrzymania losowego agenta \texttt{FactoryAgent}} \label{one-kill}
Testowanie mechanizmu naprawczego agenta zostało przeprowadzane dla systemu z trzema agentami. Po określonym czasie losowany został agent, który został zatrzymywany. Manager zanim stworzy kolejnego agenta oraz zleci mu zadanie, w celu upewnienia się, ponownie zatrzymuje poprzedniego agenta. Przedstawione poniżej logi przedstawiają przebieg mechanizmu naprawczego. Można zauważyć, że po zatrzymaniu agenta występuje problem z dostarczeniem wiadomości. Pozostali agenci oczekują na wiadomość od zatrzymanego agenta w stanie \texttt{StateWaitForProposals}.
\begin{lstlisting}
[manager@localhost: 2020-06-14 15:25:30.105538]  killing agent 2
...
[AgentC: 2020-06-14 15:25:30.253194] Finishing at state STATE_PROPOSE
...
[AgentA: 2020-06-14 15:25:30.254396] Starting negotiation
No behaviour matched for message: <message to="agentc@localhost" from="agenta@localhost" thread="None" metadata={'conversation-id': 'states', 'performative': 'propose', 'language': 'list'}>
[14, 14, 14, 14, 14, 14, 10, 10, 10, 10, 10, 2, 2, 2, 5, 5, 5, 5, 5, 5, 5, 5, 3, 3, 3, 6, 6, 6, 6, 6, 6, 0, 0, 18, 26, 26, 26, 26]
</message>
[AgentB: 2020-06-14 15:25:30.255019] Starting negotiation
No behaviour matched for message: <message to="agentc@localhost" from="agentb@localhost" thread="None" metadata={'conversation-id': 'states', 'performative': 'propose', 'language': 'list'}>
[26, 26, 26, 26, 6, 6, 6, 6, 6, 6, 3, 3, 3, 5, 5, 5, 5, 5, 5, 5, 5, 14, 14, 14, 14, 14, 14, 2, 2, 2, 0, 0, 18, 10, 10, 10, 10, 10]
</message>
...
[manager@localhost: 2020-06-14 15:25:35.210365] Pinging agentc@localhost
No behaviour matched for message: <message to="agentc@localhost" from="manager@localhost" thread="None" metadata={'conversation-id': 'watchdog', 'performative': 'inform'}>
0
</message>
...
[manager@localhost: 2020-06-14 15:26:15.210182] Pinging agentc@localhost
No behaviour matched for message: <message to="agentc@localhost" from="manager@localhost" thread="None" metadata={'conversation-id': 'watchdog', 'performative': 'inform'}>
2
</message>
...
[manager@localhost: 2020-06-14 15:26:17.212606] No response!
[manager@localhost: 2020-06-14 15:26:17.212838] Adding agentc@localhost to the team
[AgentC: 2020-06-14 15:26:17.213241] init done
[AgentC: 2020-06-14 15:26:17.255006] Setting up agent behaviours
[AgentC: 2020-06-14 15:26:17.255696] All behaviours set up
...
[AgentC: 2020-06-14 15:26:22.918378] My seq [26, 26, 26, 26, 2, 2, 2, 5, 5, 5, 5, 5, 5, 5, 5, 14, 14, 14, 14, 14, 14, 10, 10, 10, 10, 10, 18, 0, 0, 3, 3, 3, 6, 6, 6, 6, 6, 6]
[AgentC: 2020-06-14 15:26:22.918467] Starting negotiation
\end{lstlisting}
\subsection{Testowanie mechanizmu naprawczego w przypadku zatrzymania losowej liczby losowych agentów \texttt{FactoryAgent}}
W przypadku testowania mechanizmu naprawczego, kiedy zatrzymywana jest losowa liczba agentów, sposób obsługi błędu jest analogiczny do tego opisanego w punkcie \ref{one-kill}. W przedstawionym poniżej przypadku zostały zatrzymane AgentA i AgentB
\begin{lstlisting}
[manager@localhost: 2020-06-14 15:39:56.109117]  killing agent 1
[manager@localhost: 2020-06-14 15:39:56.115072]  killing agent 0
...
[AgentA: 2020-06-14 15:39:56.262956] Finishing at state STATE_PROPOSE
[AgentB: 2020-06-14 15:39:56.265086] Finishing at state STATE_PROPOSE
...
No behaviour matched for message: <message to="agenta@localhost" from="agentc@localhost" thread="None" metadata={'conversation-id': 'states', 'performative': 'propose', 'language': 'list'}>
[18, 0, 0, 6, 6, 6, 6, 6, 6, 3, 3, 3, 5, 5, 5, 5, 5, 5, 5, 5, 2, 2, 2, 26, 26, 26, 26, 14, 14, 14, 14, 14, 14, 10, 10, 10, 10, 10]
</message>
No behaviour matched for message: <message to="agentb@localhost" from="agentc@localhost" thread="None" metadata={'conversation-id': 'states', 'performative': 'propose', 'language': 'list'}>
[18, 0, 0, 6, 6, 6, 6, 6, 6, 3, 3, 3, 5, 5, 5, 5, 5, 5, 5, 5, 2, 2, 2, 26, 26, 26, 26, 14, 14, 14, 14, 14, 14, 10, 10, 10, 10, 10]
</message>
...
[manager@localhost: 2020-06-14 15:40:01.166436] Pinging agenta@localhost
No behaviour matched for message: <message to="agenta@localhost" from="manager@localhost" thread="None" metadata={'conversation-id': 'watchdog', 'performative': 'inform'}>
0
</message>
[manager@localhost: 2020-06-14 15:40:01.166858] Pinging agentb@localhost
No behaviour matched for message: <message to="agentb@localhost" from="manager@localhost" thread="None" metadata={'conversation-id': 'watchdog', 'performative': 'inform'}>
0
</message>
...
[manager@localhost: 2020-06-14 15:40:03.171134] No response!
[manager@localhost: 2020-06-14 15:40:05.173681] No response!
...
[manager@localhost: 2020-06-14 15:40:41.166766] Pinging agenta@localhost
No behaviour matched for message: <message to="agenta@localhost" from="manager@localhost" thread="None" metadata={'conversation-id': 'watchdog', 'performative': 'inform'}>
2
</message>
[manager@localhost: 2020-06-14 15:40:41.167190] Pinging agentb@localhost
No behaviour matched for message: <message to="agentb@localhost" from="manager@localhost" thread="None" metadata={'conversation-id': 'watchdog', 'performative': 'inform'}>
2
</message>
[manager@localhost: 2020-06-14 15:40:41.167518] Pinging agentc@localhost
[AgentC: 2020-06-14 15:40:41.167963] Got message
[manager@localhost: 2020-06-14 15:40:41.168377] Ping responded!
[manager@localhost: 2020-06-14 15:40:43.169765] No response!
[manager@localhost: 2020-06-14 15:40:45.170964] No response!
[manager@localhost: 2020-06-14 15:40:45.171108] Adding agenta@localhost to the team
[AgentA: 2020-06-14 15:40:45.171974] init done
[AgentA: 2020-06-14 15:40:45.212212] Setting up agent behaviours
...
[manager@localhost: 2020-06-14 15:40:45.214611] Adding agentb@localhost to the team
[AgentB: 2020-06-14 15:40:45.214894] init done
[AgentB: 2020-06-14 15:40:45.307737] Setting up agent behaviours
\end{lstlisting}
\subsection{Testowanie mechanizmu naprawczego w przypadku zatrzymania wszystkich agentów \texttt{FactoryAgent}}
System jest odporny także na zatrzymanie wszystkich agentów. Manager wykryje taki błąd analogicznie jak jest to opisane w punkcie \ref{one-kill}. Logi z przypadku testowego podczas którego zatrzymywani są wszystkie agenty zostały przedstawione poniżej.
\begin{lstlisting}
[manager@localhost: 2020-06-14 15:47:33.816601]  killing agent 0
[manager@localhost: 2020-06-14 15:47:33.816721]  killing agent 1
[manager@localhost: 2020-06-14 15:47:33.816767]  killing agent 2
...
[AgentA: 2020-06-14 15:47:33.875307] Finishing at state STATE_PROPOSE
[AgentB: 2020-06-14 15:47:33.876473] Finishing at state STATE_PROPOSE
[AgentC: 2020-06-14 15:47:33.876532] Finishing at state STATE_PROPOSE
[manager@localhost: 2020-06-14 15:47:38.877764] Pinging agenta@localhost
No behaviour matched for message: <message to="agenta@localhost" from="manager@localhost" thread="None" metadata={'conversation-id': 'watchdog', 'performative': 'inform'}>
0
</message>
[manager@localhost: 2020-06-14 15:47:38.878270] Pinging agentb@localhost
No behaviour matched for message: <message to="agentb@localhost" from="manager@localhost" thread="None" metadata={'conversation-id': 'watchdog', 'performative': 'inform'}>
0
</message>
[manager@localhost: 2020-06-14 15:47:38.878605] Pinging agentc@localhost
No behaviour matched for message: <message to="agentc@localhost" from="manager@localhost" thread="None" metadata={'conversation-id': 'watchdog', 'performative': 'inform'}>
0
</message>
[manager@localhost: 2020-06-14 15:47:40.881118] No response!
[manager@localhost: 2020-06-14 15:47:42.883007] No response!
[manager@localhost: 2020-06-14 15:47:44.885855] No response!
...
[manager@localhost: 2020-06-14 15:48:18.878783] Pinging agenta@localhost
No behaviour matched for message: <message to="agenta@localhost" from="manager@localhost" thread="None" metadata={'conversation-id': 'watchdog', 'performative': 'inform'}>
2
</message>
[manager@localhost: 2020-06-14 15:48:18.879208] Pinging agentb@localhost
No behaviour matched for message: <message to="agentb@localhost" from="manager@localhost" thread="None" metadata={'conversation-id': 'watchdog', 'performative': 'inform'}>
2
</message>
[manager@localhost: 2020-06-14 15:48:18.879481] Pinging agentc@localhost
No behaviour matched for message: <message to="agentc@localhost" from="manager@localhost" thread="None" metadata={'conversation-id': 'watchdog', 'performative': 'inform'}>
2
</message>
[manager@localhost: 2020-06-14 15:48:20.882564] No response!
[manager@localhost: 2020-06-14 15:48:22.883622] No response!
[manager@localhost: 2020-06-14 15:48:24.885824] No response!
[manager@localhost: 2020-06-14 15:48:24.885949] Adding agenta@localhost to the team
[AgentA: 2020-06-14 15:48:24.886344] init done
[AgentA: 2020-06-14 15:48:24.967628] Setting up agent behaviours
[AgentA: 2020-06-14 15:48:24.968506] All behaviours set up
...
[manager@localhost: 2020-06-14 15:48:24.969393] Task sent to agenta@localhost
[manager@localhost: 2020-06-14 15:48:24.969466] Adding agentb@localhost to the team
[AgentB: 2020-06-14 15:48:24.969950] init done
[AgentB: 2020-06-14 15:48:25.038489] Setting up agent behaviours
[AgentB: 2020-06-14 15:48:25.039391] All behaviours set up
...
[manager@localhost: 2020-06-14 15:48:25.040640] Adding agentc@localhost to the team
[AgentC: 2020-06-14 15:48:25.041053] init done
[AgentC: 2020-06-14 15:48:25.110181] Setting up agent behaviours
[AgentC: 2020-06-14 15:48:25.111143] All behaviours set up

\end{lstlisting}

\section{Podsumowanie i wnioski}
\label{sec:podsumowanie}
\todo{krótkie podsumowanie, wnioski i przemyślenia}
Podczas projektu udało się zaimplementować algorytm pozwalający na rozwiązanie problemu minimalizacji kosztu globalnego za pomocą zespołu agentów minimalizujących lokalnie koszty. Algorytm nie daje gwarancji znalezienia optymalnego globalnie rozwiązania, ale dla ograniczonych wektorów wejściowych pozwala na uzyskanie rezultatu w racjonalnym czasie.

Algorytm zaproponowany przez autorów~\cite{wooldridge1996production} nie sprawdza się dla dużych zbiorów B0'. Już dla zbioru 10 nierozróżnialnych typów aut dla agenta jego zbiór B0' przekracza $10!$ co stanowi znaczne ograniczenie zastosowania. 

Okazało się, że w przypadku mechanizmów naprawczych, znacznie większą rolę ma \texttt{ManagerAgent}, który wykrywa błąd w zachowaniu po maksymalnie 60. sekundach. W przypadku automatu stanów, obliczanie zbioru B0 dla dużych zbiorów B0' trwa dłużej, z tego powodu, by umożliwić działanie algorytmu należało zwiększyć limit.

\todo{cos o tym, ze to sa preto optymalnie i ze zawsze jest najlepiej dla jednego ziomala i ze to jest tak, zeby wyrownac straty srednio}

Wykorzystanie biblioteki \texttt{SPADE} nie było najlepszą decyzją projektową. Okazało się, że nie wspiera ona tak podstawowych akcji jak na przykład resetowanie agentów i nie jest najwygodniejsza w użyciu. \texttt{SPADE} jest cały czas rozwijaną platformą, o czym może świadczyć fakt, że najnowsza wersja pochodzi z 22.05.2020r.


\bibliographystyle{acm}
\bibliography{bibliography}

\end{document}