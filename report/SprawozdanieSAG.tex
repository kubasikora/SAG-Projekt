\documentclass{article}
\pdfpagewidth=8.5in
\pdfpageheight=11in

\usepackage{SAGreport}
% Use the postscript times font!
\usepackage{times}
\usepackage{soul}
\usepackage{url}
\usepackage{xcolor}
\usepackage{polski}
\usepackage[polish]{babel}
\usepackage[utf8]{inputenc}
\usepackage[T1]{fontenc}
\usepackage[utf8]{luainputenc}
\usepackage[hidelinks]{hyperref}
\usepackage[utf8]{inputenc}
\usepackage{caption}
\usepackage{indentfirst}
\usepackage{graphicx}
\usepackage{amsmath}
\usepackage{siunitx}
\usepackage{booktabs}
\usepackage{subfig}

\urlstyle{same}
	
\title{Systemy agentowe\\ Planowanie pracy w fabryce}

\author{
Maria Konieczka, Alicja Poturała, Jakub Sikora
\affiliations
numery albumów: 283410, 283415, 283418 \\
\emails
maria.konieczka.stud@pw.edu.pl, alicja.poturala.stud@pw.edu.pl, jakub.sikora2.stud@pw.edu.pl
}

\newcommand{\todo}[1]{\textcolor{blue}{\textbf{TO DO:} #1}}
\newcommand{\opracowanyartykul}[1]{\textcolor{teal}{\textbf{Artykuł:} #1\\}}

\begin{document}
\maketitle

\section{Wprowadzenie}
\label{sec:wprowadzenie}
\todo{napisać o problemie, o co w nim chodzi, dlaczego trzeba go rozwiazac, moze cos o licznosci zbioru ktory trzeba przeszukac}

Problem zaplanowania pracy w fabryce można zdefiniować na wiele sposobów. Wynika to między innymi z tego, że algorytm powinien być dostosowany do konkretnego procesu i obiektu. Ponadto problem ten można zdefiniować na różnych poziomach zarządzania wytwórnią. Z jednej strony, na podstawie analizy zapotrzebowania rynku, menadżerowie powinni zdecydować o liczbie konkretnych produktów do stworzenia. Z drugiej strony, należy zaplanować w jaki sposób należy zrealizować przedstawione przez zarządców zadanie.

W ramach projektu zdecydowaliśmy się na znalezieniu pareto-optymalnego rozwiązania problemu stworzenia planu realizującego zadanie postawione przez menadżera. 

Każdego dnia pracy fabryki należy przygotować konkretną sekwencję produktów, jakie będą tworzone danego dnia. Zadaniem systemu jest zaproponowanie takiej sekwencji, która jednocześnie spełniłaby postawione przez menadżera zadanie (ile poszczególnych produktów należy wytworzyć) oraz minimalizowałaby sumaryczny koszt stworzenia obiektów. Założyliśmy, że każdy produkt opisany jest za pomocą $n$ cech. Każda z cech jest nadawana produktowi w oddzielnych komórkach linii produkcyjnej. Komórki te opisywane są za pomocą zbioru wartości cech, kosztu wytworzenia danej cechy oraz kosztu dostosowania stacji do zmiany cechy.

W ramach projektu postanowiliśmy zaimplementować algorytm planujący pracę w fabryce samochodów. Zauważyliśmy, że już w przypadku zdefiniowania tylko trzech komórek fabryki (lakiernia, spawalnia i montażownia) i określeniu dla każdej cechy zbioru trzech dopuszczalnych wartości wykorzystanie rozproszonego planowania wprowadza znaczne zmniejszenie złożoności problemu. 

Sam problem znalezienia optymalnej sekwencji w sposób klasyczny sprowadza się do przeszukania przestrzeni możliwych rozwiązań i znalezienia sekwencji minimalizującej koszty dla całej fabryki. Już dla trzech komórek fabryki i trzech dopuszczalnych wartości dla każdej z cech, zbiór możliwych rozwiązań ma wymiar $27!$. Jest to problem np trudny, dla którego konieczne jest znalezienie alternatywy do klasycznych algorytmów przeszukiwania przestrzeni. Wprowadzenie konceptu agentów, którzy będą rozwiązywali problem lokalnie (z perspektywy poszczególnych komórek fabryki) pozwoli znacznie zaoszczędzić potrzebne zasoby pamięciowe oraz obliczeniowe.

\section{Zarys algorytmu}
\label{sec:algorytm}
\todo{przedstawić algorytm z~\cite{wooldridge1996production} negocjacji pomiędzy agentami, generalnie przepisać to z artykułu plus może wstawić jakiś schemat blokowy?}

\section{Architektura rozwiązania}
\label{sec:architektura}
\todo{opisać architekture rozwiazania w spade~\cite{spade}, podział systemu na agenty, agenty na zachowania, zachowania na stany, dodatkowo sekwencje komunikacji pomiedzy nimi}

\section{Mechanizmy odpornościowe}
\label{sec:recovery}
\todo{opisać mechanizm recovery na który się zdecydujemy, manager kontroluje agenty, jak on padnie to ktorys z nich przejmuje zadanie}

Dla systemów agentowych kluczowym zagadnieniem jest stworzenie mechanizmów naprawczych. Należy założyć, że zarówno komunikacja pomiędzy agentami, jak i agenci jako procesy mogą ulec awarii. Sytuacje takie należy przede wszystkim wykryć oraz, jeśli jest to możliwe, należy naprawić problem. 

\subsection{Założenia}

Dla zaproponowanej przez nas architektury rozwiązania należy zdefiniować wszystkie możliwe rodzaje awarii oraz ustalić z jakimi system będzie w stanie sobie poradzić. W rzeczywistości nie jest możliwe stworzenie mechanizmów uodparniających na wszystkie rodzaje awarii. 

W przypadku wykorzystania biblioteki \texttt{SPADE} zauważone zostało, że należy wyróżnić dwa rodzaje awarii:
\begin{enumerate}
	\item awarię pojedynczego zachowania agenta,
	\item awarię całego agenta.
\end{enumerate}

Postanowiliśmy, że kluczowe z punktu widzenia algorytmu są agenty \texttt{FactoryAgent}. Z tego powodu nie zostały stworzone zachowania odpornościowe dla agenta Menadżera a sam \texttt{ManagerAgent} odgrywa istotną rolę w mechanizmach odpornościowych agentów \texttt{FactoryAgent}. 

Aby zrealizować mechanizmy odpornościowe zaprojektowano dwa zachowania: \texttt{ControlSubordinatesBehaviour} w \texttt{ManagerAgent} oraz \texttt{WatchdogBehaviour} w \texttt{FactoryAgent}.
 
Zaimplementowano mechanizmy naprawcze dla następujących sytuacji:
\begin{itemize}
	\item wykrycie zabitego zachowania w \texttt{FactoryAgent}.
	\item brak odpowiedzi na wiadomości od współpracowników.
	\item brak odpowiedzi na wiadomość od \texttt{ManagerAgent}
\end{itemize}

Ponadto system pozwala na wykrycie awarii zachowania \texttt{ControlSubordinatesBehaviour}, czyli awarii \texttt{ManagerAgent}.

\subsection{Zaimplementowane mechanizmy}
Zadaniem \texttt{ControlSubordinatesBehaviour} jest monitorowanie działania poszczególnych agentów \texttt{FactoryAgent}. Jest to zachowanie cykliczne, które z określoną częstotliwością wysyła wiadomość typu \texttt{WatchdogMessage} do zachowań \texttt{WatchdogBehaviour} wszystkich podwładnych agentów. Zachowanie to oczekuje na odpowiedź z \texttt{WorkingState}. 

Enum \texttt{WorkingState} ma zdefiniowane trzy wartości: OK, RESTARTING oraz COMPLAINT. Pierwsze dwie wysyłane są do Managera w odpowiedzi na \texttt{WatchdogMessage}, ostatnie wysyłane jest przez pozostałe zachowania agentów, jeśli nie będą one w stanie skontaktować się z współpracownikami.

Zachowanie \texttt{ControlSubordinatesBehaviour} zlicza ile razy pod rząd nie uzyskano odpowiedzi od poszczególnych podwładnych agentów lub uzyskano odpowiedź RESTARTING. Jeśli okaże się, że agenci nie odpowiadali, lub próbowali zrestartować swoje zachowania ponad dwa razy pod rząd, manager restartuje agenta.







\section{Skalowanie rozwiązania}
\label{sec:skalowanie}
Najtrudniejszym i najbardziej wymagającym obliczeniowo elementem w \texttt{NegotiateFSM} jest wyznaczanie zbioru B0. Zbiór B0' składa się z wszystkich sekwencji dla danego agenta, które pozwalają na zminimalizowanie jego kosztu. Zbiór B0 zawiera natomiast te elementy, które mają najmniejszy sumaryczny koszt dla całej fabryki spośród zbioru B0'.

Aby wyznaczyć zbiór B0 należy więc dla wszystkich sekwencji z B0' wysłać wiadomości do wszystkich współpracowników. W całym systemie, na tym etapie negocjacji wysyłanych jest $(n-1)\sum_{i \in Ag}|B0'_i|$ wiadomości, gdzie $n$ to liczba agentów \texttt{FactoryAgent} w systemie. 

Wbrew pozorom, podczas przeprowadzania testów, okazało się, że wielkość zbioru B0' odgrywa większą rolę na szybkość wyznaczania rozwiązania, niż liczba agentów. Wielkość zbioru B0' dla każdego agenta zależy wykładniczo od liczności nierozróżnialnych dla danego agenta produktów do wyprodukowania. Wyznaczenie B0' sprowadza się do obliczenia permutacji wewnątrz nierozróżnialnych zbiorów a następnie wyznaczenia wszystkich możliwych kombinacji połączeń tych zbiorów. 

Liczba nierozróżnialnych dla danego agenta produktów zależy od liczby typów produktów zleconych do wykonania. Dla agenta, który odpowiada za cechę, która może mieć trzy wartości, w przypadku zlecenia 12 różnych produktów, w najkorzystniejszym przypadku, liczność zbioru B0' wynosi $4!\cdot 4!\cdot 4!\cdot 6$ czyli $\num{82944}$, natomiast dla 11 różnych produktów $3!\cdot 4!\cdot 4!\cdot 6$ czyli $\num{20736}$.

Z tego powodu postanowiliśmy ograniczyć liczbę typów do 9. Wykorzystanie algorytmu jest nadal opłacalne, gdyż przeszukiwany zbiór rozwiązań jest mniejszy o kilka rzędów wielkości niż $9!$. Jeśli chodzi o liczbę wiadomości wysyłanych w systemie, to w ogólności zależy ona od kwadratu liczby agentów (dokładnie $n \cdot (n-1)$)



\section{Podsumowanie i wnioski}
\label{sec:podsumowanie}
\todo{krótkie podsumowanie, wnioski i przemyślenia}

\bibliographystyle{acm}
\bibliography{bibliography}

\end{document}