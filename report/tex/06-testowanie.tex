System został przetestowany dla siedmiu przypadków. Do testowania wykorzystano maszynę wirtualną z system Ubuntu 18.04. Dla każdego z przypadków, założono, że każdy agent odpowiada za cechę, która może mieć trzy wartości.
\subsection{Testowanie dla 3 agentów}
W przypadku testowania systemu dla 3 agentów, wykorzystano plik konfiguracyjny ze zleceniem wejściowym \textit{generate\_vectore\_agent3.json}. Rozmiar zbioru B0' dla poszczególnych agentów wynosił 3456, 2880 i 1728. Agentom udało się dojść do porozumienia i wyznaczyć rozwiązanie po niecałych 29. sekundach.
\begin{lstlisting}
[manager@localhost: 2020-06-14 13:15:42.700852] DeputeBehaviour for agentc@localhost created
[manager@localhost: 2020-06-14 13:15:42.701610] Task sent to agenta@localhost
[manager@localhost: 2020-06-14 13:15:42.702028] Task sent to agentb@localhost
[manager@localhost: 2020-06-14 13:15:42.702429] Task sent to agentc@localhost
[manager@localhost: 2020-06-14 13:15:42.702685] Every task sent
...
[manager@localhost: 2020-06-14 13:16:11.452603] Got optimal sequence [26, 26, 26, 26, 6, 6, 6, 6, 6, 6, 3, 3, 3, 5, 5, 5, 5, 5, 5, 5, 5, 14, 14, 14, 14, 14, 14, 10, 10, 10, 10, 10, 18, 0, 0, 2, 2, 2] with total cost of 625
[main: 2020-06-14 13:16:11.673955] Agents finished
\end{lstlisting}

\subsection{Testowanie dla 8 agentów}
W przypadku testowania systemu dla 8 agentów, wykorzystano plik konfiguracyjny ze zleceniem wejściowym \textit{generate\_vectore\_agent8.json}. Rozmiar zbioru B0' dla poszczególnych agentów wynosił 2880, 1728, 1728, 1728, 3456, 1728, 5760 i 8640. Agentom udało się dojść do porozumienia i wyznaczyć rozwiązanie po około $\num{13,5}$ minutach.
\begin{lstlisting}
[manager@localhost: 2020-06-14 13:40:20.139839] Task sent to agenta@localhost
[manager@localhost: 2020-06-14 13:40:20.140737] Task sent to agentb@localhost
[manager@localhost: 2020-06-14 13:40:20.141691] Task sent to agentc@localhost
[manager@localhost: 2020-06-14 13:40:20.142648] Task sent to agentd@localhost
[manager@localhost: 2020-06-14 13:40:20.143593] Task sent to agente@localhost
[manager@localhost: 2020-06-14 13:40:20.145955] Task sent to agentf@localhost
[manager@localhost: 2020-06-14 13:40:20.147695] Task sent to agentg@localhost
[manager@localhost: 2020-06-14 13:40:20.148860] Task sent to agenth@localhost
[manager@localhost: 2020-06-14 13:40:20.150821] Every task sent
...
[AgentA: 2020-06-14 13:53:47.057112] Agreement found!!
[AgentA: 2020-06-14 13:53:47.057171] The end
[AgentA: 2020-06-14 13:53:47.057211] The cost 1820
[AgentA: 2020-06-14 13:53:47.057246] Winning sequence [686, 1718, 1718, 1718, 1718, 1718, 1029, 1029, 1029, 1029, 1029, 1029, 1029, 1029, 3307, 3307, 3307, 3307, 3307, 3307, 3842, 3842, 3842, 2597, 2597, 2597, 2597, 2597, 2597, 2192, 2192, 2192, 2192, 2192, 2192, 2192, 2192, 4432, 6492, 6492, 6492]
[manager@localhost: 2020-06-14 13:53:47.057812] Got optimal sequence [686, 1718, 1718, 1718, 1718, 1718, 1029, 1029, 1029, 1029, 1029, 1029, 1029, 1029, 3307, 3307, 3307, 3307, 3307, 3307, 3842, 3842, 3842, 2597, 2597, 2597, 2597, 2597, 2597, 2192, 2192, 2192, 2192, 2192, 2192, 2192, 2192, 4432, 6492, 6492, 6492] with total cost of 1820
\end{lstlisting}
\subsection{Testowanie dla 10 agentów}
W przypadku testowania systemu dla 10 agentów, wykorzystano plik konfiguracyjny ze zleceniem wejściowym \textit{generate\_vectore\_agent10.json}. Rozmiar zbioru B0' dla poszczególnych agentów wynosił 2880, 1728, 1296, 1728, 4320, 1728, 1296, 4320, 1728, 1728. Agentom udało się dojść do porozumienia i wyznaczyć rozwiązanie po około 33 minutach.
\begin{lstlisting}
[manager@localhost: 2020-06-14 13:58:58.970668] DeputeBehaviour for agentj@localhost created
[manager@localhost: 2020-06-14 13:58:58.982009] Task sent to agenta@localhost
[manager@localhost: 2020-06-14 13:58:58.996478] Task sent to agentb@localhost
[manager@localhost: 2020-06-14 13:58:59.004003] Task sent to agentc@localhost
[manager@localhost: 2020-06-14 13:58:59.018002] Task sent to agentd@localhost
[manager@localhost: 2020-06-14 13:58:59.024701] Task sent to agente@localhost
[manager@localhost: 2020-06-14 13:58:59.031039] Task sent to agentf@localhost
[manager@localhost: 2020-06-14 13:58:59.037846] Task sent to agentg@localhost
[manager@localhost: 2020-06-14 13:58:59.043934] Task sent to agenth@localhost
[manager@localhost: 2020-06-14 13:58:59.050280] Task sent to agenti@localhost
[manager@localhost: 2020-06-14 13:58:59.056439] Task sent to agentj@localhost
[manager@localhost: 2020-06-14 13:58:59.056495] Every task sent
...
[manager@localhost: 2020-06-14 14:31:54.287116] Got optimal sequence [7616, 7616, 7616, 7616, 28274, 28274, 47912, 47912, 47912, 47912, 38679, 38679, 38679, 38679, 4609, 4609, 4609, 4609, 4609, 15489, 43198, 43198, 43198, 23900, 35267, 35267, 35267, 35267, 35267, 35267, 35267] with total cost of 1760
[main: 2020-06-14 14:31:54.721081] Agents finished
\end{lstlisting}
\subsection{Testowanie mechanizmu naprawczego zachowania}
Testowanie mechanizmu naprawczego zachowanie zostało przeprowadzane dla systemu z trzema agentami. Po określonym czasie losowany został agent oraz jego zachowanie, które to następnie zostało zabijane. Restartowanie samego zachowania jest błyskawiczne, dlatego można zauważyć, że już w następnej iteracji zachowania \texttt{ControlSubordinatesBehaviour} zachowanie działa poprawnie. W przedstawionym poniżej teście wylosowany został AgentB oraz zachowanie \texttt{ComputeAgentsCost}. W logach można zauważyć, że o 14:49:33 zostało zabite zachowanie oraz, że wiadomości wysłane do tego zachowania nie zostały dostarczone. Ponad to, można zauważyć, że AgentB wykrył niedziałające zachowanie o 14:49:48. Od tego czasu pozostali agenci nie mieli problemu z nawiązaniem kontaktu, o czym może świadczyć fakt błyskawicznego przejścia do kolejnego stanu agentów AgentA i AgentC.
\begin{lstlisting}
[manager@localhost: 2020-06-14 14:49:33.063398]  killing agent 1 behaviour 
...
No behaviour matched for message: <message to="agentb@localhost" from="agenta@localhost" thread="None" metadata={'conversation-id': 'cost', 'performative': 'request', 'save': 'True'}>
[10, 10, 10, 10, 10, 14, 14, 14, 14, 14, 14, 6, 6, 6, 6, 6, 6, 2, 2, 2, 3, 3, 3, 5, 5, 5, 5, 5, 5, 5, 5, 0, 0, 26, 26, 26, 26, 18]
</message>
No behaviour matched for message: <message to="agentb@localhost" from="agentc@localhost" thread="None" metadata={'conversation-id': 'cost', 'performative': 'request', 'save': 'True'}>
[10, 10, 10, 10, 10, 6, 6, 6, 6, 6, 6, 18, 3, 3, 3, 0, 0, 14, 14, 14, 14, 14, 14, 2, 2, 2, 26, 26, 26, 26, 5, 5, 5, 5, 5, 5, 5, 5]
</message>
...
[AgentA: 2020-06-14 14:49:51.796292] Starting negotiation
...
[AgentC: 2020-06-14 14:49:52.773619] Starting negotiation
...
[manager@localhost: 2020-06-14 14:49:48.103538] Pinging agentb@localhost
...
[AgentB: 2020-06-14 14:49:48.104472] Got message
[AgentB: 2020-06-14 14:49:48.104609] Restarting
...
[manager@localhost: 2020-06-14 14:50:08.121212] Pinging agentb@localhost
[manager@localhost: 2020-06-14 14:50:08.122911] Ping responded!
\end{lstlisting}
\subsection{Testowanie mechanizmu naprawczego w przypadku zabicia losowego agenta \texttt{FactoryAgent}} \label{one-kill}
Testowanie mechanizmu naprawczego agenta zostało przeprowadzane dla systemu z trzema agentami. Po określonym czasie losowany został agent, który został zatrzymywany. Manager zanim stworzy kolejnego agenta oraz zleci mu zadanie, w celu upewnienia się, ponownie zatrzymuje poprzedniego agenta. Przedstawione poniżej logi przedstawiają przebieg mechanizmu naprawczego. Można zauważyć, że po zatrzymaniu agenta występuje problem z dostarczeniem wiadomości. Pozostali agenci oczekują na wiadomość od zatrzymanego agenta w stanie \texttt{StateWaitForProposals}.
\begin{lstlisting}
[manager@localhost: 2020-06-14 15:25:30.105538]  killing agent 2
...
[AgentC: 2020-06-14 15:25:30.253194] Finishing at state STATE_PROPOSE
...
[AgentA: 2020-06-14 15:25:30.254396] Starting negotiation
No behaviour matched for message: <message to="agentc@localhost" from="agenta@localhost" thread="None" metadata={'conversation-id': 'states', 'performative': 'propose', 'language': 'list'}>
[14, 14, 14, 14, 14, 14, 10, 10, 10, 10, 10, 2, 2, 2, 5, 5, 5, 5, 5, 5, 5, 5, 3, 3, 3, 6, 6, 6, 6, 6, 6, 0, 0, 18, 26, 26, 26, 26]
</message>
[AgentB: 2020-06-14 15:25:30.255019] Starting negotiation
No behaviour matched for message: <message to="agentc@localhost" from="agentb@localhost" thread="None" metadata={'conversation-id': 'states', 'performative': 'propose', 'language': 'list'}>
[26, 26, 26, 26, 6, 6, 6, 6, 6, 6, 3, 3, 3, 5, 5, 5, 5, 5, 5, 5, 5, 14, 14, 14, 14, 14, 14, 2, 2, 2, 0, 0, 18, 10, 10, 10, 10, 10]
</message>
...
[manager@localhost: 2020-06-14 15:25:35.210365] Pinging agentc@localhost
No behaviour matched for message: <message to="agentc@localhost" from="manager@localhost" thread="None" metadata={'conversation-id': 'watchdog', 'performative': 'inform'}>
0
</message>
...
[manager@localhost: 2020-06-14 15:26:15.210182] Pinging agentc@localhost
No behaviour matched for message: <message to="agentc@localhost" from="manager@localhost" thread="None" metadata={'conversation-id': 'watchdog', 'performative': 'inform'}>
2
</message>
...
[manager@localhost: 2020-06-14 15:26:17.212606] No response!
[manager@localhost: 2020-06-14 15:26:17.212838] Adding agentc@localhost to the team
[AgentC: 2020-06-14 15:26:17.213241] init done
[AgentC: 2020-06-14 15:26:17.255006] Setting up agent behaviours
[AgentC: 2020-06-14 15:26:17.255696] All behaviours set up
...
[AgentC: 2020-06-14 15:26:22.918378] My seq [26, 26, 26, 26, 2, 2, 2, 5, 5, 5, 5, 5, 5, 5, 5, 14, 14, 14, 14, 14, 14, 10, 10, 10, 10, 10, 18, 0, 0, 3, 3, 3, 6, 6, 6, 6, 6, 6]
[AgentC: 2020-06-14 15:26:22.918467] Starting negotiation
\end{lstlisting}
\subsection{Testowanie mechanizmu naprawczego w przypadku zabicia losowej liczby losowych agentów \texttt{FactoryAgent}}
W przypadku testowania mechanizmu naprawczego, kiedy zatrzymywana jest losowa liczba agentów, sposób obsługi błędu jest analogiczny do tego opisanego w punkcie \ref{one-kill}. W przedstawionym poniżej przypadku zostały zatrzymane AgentA i AgentB
\begin{lstlisting}
[manager@localhost: 2020-06-14 15:39:56.109117]  killing agent 1
[manager@localhost: 2020-06-14 15:39:56.115072]  killing agent 0
...
[AgentA: 2020-06-14 15:39:56.262956] Finishing at state STATE_PROPOSE
[AgentB: 2020-06-14 15:39:56.265086] Finishing at state STATE_PROPOSE
...
No behaviour matched for message: <message to="agenta@localhost" from="agentc@localhost" thread="None" metadata={'conversation-id': 'states', 'performative': 'propose', 'language': 'list'}>
[18, 0, 0, 6, 6, 6, 6, 6, 6, 3, 3, 3, 5, 5, 5, 5, 5, 5, 5, 5, 2, 2, 2, 26, 26, 26, 26, 14, 14, 14, 14, 14, 14, 10, 10, 10, 10, 10]
</message>
No behaviour matched for message: <message to="agentb@localhost" from="agentc@localhost" thread="None" metadata={'conversation-id': 'states', 'performative': 'propose', 'language': 'list'}>
[18, 0, 0, 6, 6, 6, 6, 6, 6, 3, 3, 3, 5, 5, 5, 5, 5, 5, 5, 5, 2, 2, 2, 26, 26, 26, 26, 14, 14, 14, 14, 14, 14, 10, 10, 10, 10, 10]
</message>
...
[manager@localhost: 2020-06-14 15:40:01.166436] Pinging agenta@localhost
No behaviour matched for message: <message to="agenta@localhost" from="manager@localhost" thread="None" metadata={'conversation-id': 'watchdog', 'performative': 'inform'}>
0
</message>
[manager@localhost: 2020-06-14 15:40:01.166858] Pinging agentb@localhost
No behaviour matched for message: <message to="agentb@localhost" from="manager@localhost" thread="None" metadata={'conversation-id': 'watchdog', 'performative': 'inform'}>
0
</message>
...
[manager@localhost: 2020-06-14 15:40:03.171134] No response!
[manager@localhost: 2020-06-14 15:40:05.173681] No response!
...
[manager@localhost: 2020-06-14 15:40:41.166766] Pinging agenta@localhost
No behaviour matched for message: <message to="agenta@localhost" from="manager@localhost" thread="None" metadata={'conversation-id': 'watchdog', 'performative': 'inform'}>
2
</message>
[manager@localhost: 2020-06-14 15:40:41.167190] Pinging agentb@localhost
No behaviour matched for message: <message to="agentb@localhost" from="manager@localhost" thread="None" metadata={'conversation-id': 'watchdog', 'performative': 'inform'}>
2
</message>
[manager@localhost: 2020-06-14 15:40:41.167518] Pinging agentc@localhost
[AgentC: 2020-06-14 15:40:41.167963] Got message
[manager@localhost: 2020-06-14 15:40:41.168377] Ping responded!
[manager@localhost: 2020-06-14 15:40:43.169765] No response!
[manager@localhost: 2020-06-14 15:40:45.170964] No response!
[manager@localhost: 2020-06-14 15:40:45.171108] Adding agenta@localhost to the team
[AgentA: 2020-06-14 15:40:45.171974] init done
[AgentA: 2020-06-14 15:40:45.212212] Setting up agent behaviours
...
[manager@localhost: 2020-06-14 15:40:45.214611] Adding agentb@localhost to the team
[AgentB: 2020-06-14 15:40:45.214894] init done
[AgentB: 2020-06-14 15:40:45.307737] Setting up agent behaviours
\end{lstlisting}
\subsection{Testowanie mechanizmu naprawczego w przypadku zabicia wszystkich agentów \texttt{FactoryAgent}}
System jest odporny także na zatrzymanie wszystkich agentów. Manager wykryje taki błąd analogicznie jak jest to opisane w punkcie \ref{one-kill}. Logi z przypadku testowego podczas którego zatrzymywani są wszystkie agenty zostały przedstawione poniżej.
\begin{lstlisting}
[manager@localhost: 2020-06-14 15:47:33.816601]  killing agent 0
[manager@localhost: 2020-06-14 15:47:33.816721]  killing agent 1
[manager@localhost: 2020-06-14 15:47:33.816767]  killing agent 2
...
[AgentA: 2020-06-14 15:47:33.875307] Finishing at state STATE_PROPOSE
[AgentB: 2020-06-14 15:47:33.876473] Finishing at state STATE_PROPOSE
[AgentC: 2020-06-14 15:47:33.876532] Finishing at state STATE_PROPOSE
[manager@localhost: 2020-06-14 15:47:38.877764] Pinging agenta@localhost
No behaviour matched for message: <message to="agenta@localhost" from="manager@localhost" thread="None" metadata={'conversation-id': 'watchdog', 'performative': 'inform'}>
0
</message>
[manager@localhost: 2020-06-14 15:47:38.878270] Pinging agentb@localhost
No behaviour matched for message: <message to="agentb@localhost" from="manager@localhost" thread="None" metadata={'conversation-id': 'watchdog', 'performative': 'inform'}>
0
</message>
[manager@localhost: 2020-06-14 15:47:38.878605] Pinging agentc@localhost
No behaviour matched for message: <message to="agentc@localhost" from="manager@localhost" thread="None" metadata={'conversation-id': 'watchdog', 'performative': 'inform'}>
0
</message>
[manager@localhost: 2020-06-14 15:47:40.881118] No response!
[manager@localhost: 2020-06-14 15:47:42.883007] No response!
[manager@localhost: 2020-06-14 15:47:44.885855] No response!
...
[manager@localhost: 2020-06-14 15:48:18.878783] Pinging agenta@localhost
No behaviour matched for message: <message to="agenta@localhost" from="manager@localhost" thread="None" metadata={'conversation-id': 'watchdog', 'performative': 'inform'}>
2
</message>
[manager@localhost: 2020-06-14 15:48:18.879208] Pinging agentb@localhost
No behaviour matched for message: <message to="agentb@localhost" from="manager@localhost" thread="None" metadata={'conversation-id': 'watchdog', 'performative': 'inform'}>
2
</message>
[manager@localhost: 2020-06-14 15:48:18.879481] Pinging agentc@localhost
No behaviour matched for message: <message to="agentc@localhost" from="manager@localhost" thread="None" metadata={'conversation-id': 'watchdog', 'performative': 'inform'}>
2
</message>
[manager@localhost: 2020-06-14 15:48:20.882564] No response!
[manager@localhost: 2020-06-14 15:48:22.883622] No response!
[manager@localhost: 2020-06-14 15:48:24.885824] No response!
[manager@localhost: 2020-06-14 15:48:24.885949] Adding agenta@localhost to the team
[AgentA: 2020-06-14 15:48:24.886344] init done
[AgentA: 2020-06-14 15:48:24.967628] Setting up agent behaviours
[AgentA: 2020-06-14 15:48:24.968506] All behaviours set up
...
[manager@localhost: 2020-06-14 15:48:24.969393] Task sent to agenta@localhost
[manager@localhost: 2020-06-14 15:48:24.969466] Adding agentb@localhost to the team
[AgentB: 2020-06-14 15:48:24.969950] init done
[AgentB: 2020-06-14 15:48:25.038489] Setting up agent behaviours
[AgentB: 2020-06-14 15:48:25.039391] All behaviours set up
...
[manager@localhost: 2020-06-14 15:48:25.040640] Adding agentc@localhost to the team
[AgentC: 2020-06-14 15:48:25.041053] init done
[AgentC: 2020-06-14 15:48:25.110181] Setting up agent behaviours
[AgentC: 2020-06-14 15:48:25.111143] All behaviours set up

\end{lstlisting}