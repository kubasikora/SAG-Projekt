\todo{napisać o problemie, o co w nim chodzi, dlaczego trzeba go rozwiazac, moze cos o licznosci zbioru ktory trzeba przeszukac}

Problem zaplanowania pracy w fabryce można zdefiniować na wiele sposobów. Wynika to między innymi z tego, że algorytm powinien być dostosowany do konkretnego procesu i obiektu. Ponadto problem ten można zdefiniować na różnych poziomach zarządzania wytwórnią. Z jednej strony, na podstawie analizy zapotrzebowania rynku, menadżerowie powinni zdecydować o liczbie konkretnych produktów do stworzenia. Z drugiej strony, należy zaplanować w jaki sposób należy zrealizować przedstawione przez zarządców zadanie.

W ramach projektu zdecydowaliśmy się na znalezieniu pareto-optymalnego rozwiązania problemu stworzenia planu realizującego zadanie postawione przez menadżera. 

Każdego dnia pracy fabryki należy przygotować konkretną sekwencję produktów, jakie będą tworzone danego dnia. Zadaniem systemu jest zaproponowanie takiej sekwencji, która jednocześnie spełniłaby postawione przez menadżera zadanie (ile poszczególnych produktów należy wytworzyć) oraz minimalizowałaby sumaryczny koszt stworzenia obiektów. Założyliśmy, że każdy produkt opisany jest za pomocą $n$ cech. Każda z cech jest nadawana produktowi w oddzielnych komórkach linii produkcyjnej. Komórki te opisywane są za pomocą zbioru wartości cech, kosztu wytworzenia danej cechy oraz kosztu dostosowania stacji do zmiany cechy.

W ramach projektu postanowiliśmy zaimplementować algorytm planujący pracę w fabryce samochodów. Zauważyliśmy, że już w przypadku zdefiniowania tylko trzech komórek fabryki (lakiernia, spawalnia i montażownia) i określeniu dla każdej cechy zbioru trzech dopuszczalnych wartości wykorzystanie rozproszonego planowania wprowadza znaczne zmniejszenie złożoności problemu. 

Sam problem znalezienia optymalnej sekwencji w sposób klasyczny sprowadza się do przeszukania przestrzeni możliwych rozwiązań i znalezienia sekwencji minimalizującej koszty dla całej fabryki. Już dla trzech komórek fabryki i trzech dopuszczalnych wartości dla każdej z cech, zbiór możliwych rozwiązań ma wymiar $27!$. Jest to problem np trudny, dla którego konieczne jest znalezienie alternatywy do klasycznych algorytmów przeszukiwania przestrzeni. Wprowadzenie konceptu agentów, którzy będą rozwiązywali problem lokalnie (z perspektywy poszczególnych komórek fabryki) pozwoli znacznie zaoszczędzić potrzebne zasoby pamięciowe oraz obliczeniowe.