\todo{moze jakies rysunki?}

Dla systemów agentowych kluczowym zagadnieniem jest stworzenie mechanizmów naprawczych. Należy założyć, że zarówno komunikacja pomiędzy agentami, jak i agenci jako procesy mogą ulec awarii. Sytuacje takie należy przede wszystkim wykryć oraz, jeśli jest to możliwe, naprawić. 

\subsection{Założenia}

Dla zaproponowanej przez nas architektury rozwiązania zdefiniowaliśmy wszystkie możliwe rodzaje awarii oraz ustaliliśmy z jakimi system będzie w stanie sobie poradzić. W rzeczywistości nie jest możliwe stworzenie mechanizmów uodparniających na wszystkie rodzaje awarii. 

W przypadku wykorzystania biblioteki \texttt{SPADE} zauważone zostało, że należy wyróżnić dwa rodzaje awarii:
\begin{enumerate}
	\item awarię pojedynczego zachowania agenta \texttt{FactoryAgent},
	\item awarię całego agenta \texttt{FactoryAgent},
	\item awarię pojedynczego zachowania agenta \texttt{ManagerAgent},
	\item awarię całego agenta \texttt{ManagerAgent}.
\end{enumerate}

Postanowiliśmy, że kluczowe z punktu widzenia algorytmu są agenty \texttt{FactoryAgent}. Z tego powodu nie zostały stworzone zachowania odpornościowe dla agenta zarządcy a sam \texttt{ManagerAgent} odgrywa istotną rolę w mechanizmach odpornościowych agentów \texttt{FactoryAgent}. 
 
Stworzono mechanizmy naprawcze dla następujących sytuacji:
\begin{itemize}
	\item wykrycie zabitego zachowania w \texttt{FactoryAgent}.
	\item brak odpowiedzi na wiadomości od współpracowników.
	\item brak odpowiedzi na wiadomość od \texttt{ManagerAgent}
\end{itemize}

Ponadto system pozwala na wykrycie awarii zachowania \texttt{ControlSubordinatesBehaviour}, czyli awarii \texttt{ManagerAgent}.

\subsection{Zaimplementowane mechanizmy}
Aby zrealizować mechanizmy odpornościowe zaprojektowano dwa zachowania: \texttt{ControlSubordinatesBehaviour} w \texttt{ManagerAgent} oraz \texttt{WatchdogBehaviour} w \texttt{FactoryAgent}.

Zadaniem \texttt{ControlSubordinatesBehaviour} jest monitorowanie działania poszczególnych agentów \texttt{FactoryAgent}. Jest to zachowanie cykliczne, które z określoną częstotliwością wysyła wiadomość typu \texttt{WatchdogMessage} do zachowań \texttt{WatchdogBehaviour} wszystkich podwładnych agentów. Zachowanie to oczekuje na odpowiedź z \texttt{WorkingState}. 

Enum \texttt{WorkingState} ma zdefiniowane trzy wartości: \texttt{OK}, \texttt{RESTARTING} oraz \texttt{COMPLAINT}. Pierwsze dwie wysyłane są do Managera w odpowiedzi na \texttt{WatchdogMessage}, a ostatnie wysyłane jest przez pozostałe zachowania agentów, jeśli nie będą one w stanie skontaktować się z współpracownikami.

Zachowanie \texttt{ControlSubordinatesBehaviour} zlicza ile razy pod rząd nie uzyskano odpowiedzi od poszczególnych podwładnych agentów lub uzyskano odpowiedź \texttt{RESTARTING}. Jeśli okaże się, że agenci nie odpowiadali lub próbowali zrestartować swoje zachowania ponad dwa razy pod rząd, manager restartuje agenta. 

Przy tworzeniu agenta w czasie trwania negocjacji istotne jest, aby przekazać mu dwie kluczowe zmienne stanu: jego aktywność oraz aktywność współpracowników. Są to zmienne krytyczne z punktu widzenia algorytmu negocjacji. 

Ponadto, jeśli restartowany agent aktywnie brał udział w negocjacjach, nie należy doprowadzić do desynchronizacji z pozostałymi agentami. Z tego powodu, we wszystkich synchronizowanych stanach dodano przejście do stanu \texttt{STATE\_PROPOSE}. 

Agenty, jeśli wykryją, że któryś z ich współpracowników zbyt długo nie przysłał wiadomości ze swoją informacją lub decyzją wysyłają do managera wiadomość ze skargą. W treści wiadomości zapisany jest jid agenta, który nie wysłał wiadomości. Zmiana stanu jest dokonywana, by ponownie zsynchronizować agenty. \texttt{STATE\_PROPOSE} jest pierwszym stanem wymagającym synchronizacji po dokonaniu inicjalizacji.

Zachowanie \texttt{WatchdogBehaviour} agenta \texttt{FactoryAgent} jest również cykliczne. Z zadaną częstotliwością sprawdza, czy przyszła wiadomość typu \texttt{WatchdogMessage} od zarządcy. Aby odpowiedzieć, sprawdza stan zachowań będącymi serwisami agenta oraz zwraca wartość \texttt{OK} jeśli wszystkie serwisy działają poprawnie lub \texttt{RESTARTING} jeśli okazało się, że któreś zachowanie został zabite. W sprawdzaniu zachowań pomięty został \texttt{NegotiateFSM}. Uznano, że restartowanie tego zachowania jest praktycznie równoznaczne z restartowaniem całego agenta.





